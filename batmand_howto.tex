\documentclass[
	12pt,
	a4paper,
	twoside,
	english,
	headsepline,
	footnosepline,
	automark,
	normalheadings,
	openany,
	cleardoubleplain,
	abstracton,
	idxtotoc,
	liststotoc,
	bibtotoc,
 	BCOR8mm,
]{scrartcl}

\usepackage{
	calc,
	color,
	fancybox,
	fancyvrb,
	float,
	mdwlist,
	scrdate,
	scrtime,
	scrpage2,
	tabularx,
	ucs,
	multirow
}

\usepackage[utf8x]{inputenc}
\usepackage{listings}

\usepackage{graphicx}
\graphicspath{{../img/}}
\usepackage{hyperref}

\pagestyle{scrheadings}	% Standart  Kopf- und Fuï¿œzeile
\setkomafont{pagehead}{\small\scshape}
\setcounter{tocdepth}{3}

\newcommand{\dctitle}{B.A.T.M.A.N. Daemon HowTo}
\newcommand{\dcsubject}{~}
\newcommand{\dcauthorlastname}{Tsai}
\newcommand{\dcauthorfirstname}{Wesley}
\newcommand{\dcauthoremail}{wesleyboy42(AT)gmail(DOT)com}
\newcommand{\dcdate}{\today}

\hypersetup{
	pdftitle	= {\dctitle}, %
	pdfsubject	= {\dcsubject, \dcdate}, %
	pdfauthor	= {\dcauthorfirstname~\dcauthorlastname, \dcauthoremail}, %
	pdfkeywords	= {mesh}, %
	pdfcreator	= {pdfTeX with Hyperref}, %
	pdfproducer	= {LaTeX, hyperref, thumbpdf}, %
	colorlinks=false, %
	linkbordercolor= {1 1 1}, %
}
\title{\dctitle}
\author{\dcauthorfirstname~\dcauthorlastname}

\fussy
\begin{document}
\maketitle
B.A.T.M.A.N stands for ? better approach to mobile ad hoc networking ?, this is a new  routing protocol for multi hop ad hoc mesh networks. Go to http://open mesh.net to see  more information. The following document will explain how to install and use the batman daemon.


\section{Installing from source}
?

\subsection{Prerequirements}
?

Compile environment and libraries. 1. 2. 3. 4. gcc  libc6 dev build essential  binutils 5. makedev 6. make 7. libpthread

?

Download the batman daemon code from the website

 http://open mesh.net/batman/downloads
?

\subsection{Compiling}
All you have to do is untar and make, then you will see the executable file called ?batmand?. \$ wget http://downloads.open mesh.net/batman/stable/sources/batmand\_0.2  current\_sources.tgz \$ tar xzvf batmand\_0.2 current\_sources.tgz \$ cd batmand\_0.2 rv451\_sources \$ make If you want reduce the size of executable file, just strip it by executing:  \$ strip batmand Note that if you want to help us finding a bug in the daemon, please don't strip it.

\subsection{Installing}
Copy ?batmand? to a location somewhere in your path, for example \$ cp batmand /usr/sbin/ Or start it right from the directory where you compiled it \$ ./batmand

\section{Usage}
If you execute batmand  h or  H, you will see the help page, in the following we will explain  all parameters and how to work with them in a mesh network.
?

batmand [options] interface [interface interface]
?

\subsection{SYNOPSIS}

\subsection{DESCRIPTION}
You can start batmand without specifying options, but you have to choose at least one  interface batman runs on. \$ batmand eth1 If you have more interfaces, then you just add them behind the first. \$ batmand eth1 eth2 eth3 The batman daemon can also run on alias interfaces. Note that we use alias interfaces to separate batman routing protocol and olsr routing  protocol. \$ batmand eth1:test1 eth2:test2 eth3:test3 Note that the batman daemon will take the ip address and the broadcast address from the  given interfaces. Note also that you have to check whether your essid, channel or wifi mode is correct or not.
 o originator interval in ms Originator means a node transmits broadcast messages (we call them originator message or  OGM) to inform the neighboring nodes about its existence. Originator interval is the time to wait after sending one message and before the batman  daemon sends the next message. The default value is 1000 ms ( 1 second ). In a mobile network, you may want to detect network changes very quickly, so you need to  send message very often, for example, use a value of 500 ms. In a static network, you can save bandwidth by using a higher value. \$ batmand  o 2000 eth1 In this case, batmand will wait 2 second until sending the next OGMs.  d debug level The debug level can be set to five values.           default:   0  > debug disabled           allowed values:   1  > list neighbors                               2  > list gateways                               3  > observe batman                               4  > observe batman (very verbose) Level 1 just lists the neighbors in your batman network. The output looks like this Originator           Router (\#/128):    Potential routers... [B.A.T.M.A.N. 0.2, MainIF/IP:   eth2 105.131.131.175, UT: 0d 0h 3m] 105.131.83.2        105.131.1.3 (  71):     105.131.1.3 (  71) 105.131.1.2         105.131.1.2 (  52):     105.131.1.2 (  52) 105.131.56.10     105.131.1.4 (  25):   105.131.1.4 (  25),105.131.1.6 ( 15), 105.131.1.7 ( 10),   .... 105.131.131.70   105.131.131.70 (121):  105.131.131.70 (121)
?

? ?

In the first line, we will see the version of the batman daemon, main interface, main  IP, and uptime. In the first column, we can see those IPs  which we can reach. In the second column, we can see those IPs which we sent our packets to when we  want to reach the IP of the first column. The number in the parenthesis indicates the  link quality of the connection and the \#/128 shows the maximum number of packets. In the third column, we can see those IPs which are one hop neighbors and  rebroadcasted packets from the originator. The batman daemon will choose the router  with the best link quality from the potential router list.  In this case, 105.131.1.2 is a one hop neighbor of 105.131.131.175, because the 105.131.1.2  is  originator, router and potential router at the same time. If 105.131.131.175 wants to exchange  data with the 105.131.83.2, then it will sent its packets to the 105.131.1.3, because it is the  router for this destination. Level 2 just lists gateways in the batman network. The output looks like this Gateway              Router (\#/128)   105.131.83.5       105.131.41.1 ( 57), gw\_class 11   >6 MBit, reliability: 0   105.131.41.5       105.131.41.1 ( 53), gw\_class 11   >6 MBit, reliability: 0
? ?

In the first column, we can see those IPs  which are our gateways. In the second column, we can see those IPs which we sent our packets to when we  want to reach the IP of the first column. The number in the parenthesis indicates the  link quality of the connection and the \#/128 shows the maximum number of packets.  The ?gw\_class? means gateway class of the gateway and ?11  >6 MBit? means how  much bandwidth the gateway owner wants to share. The reliability means how good  the quality of the internet connection is. In this case, ?0? means this is the best  quality. The reliability number will increase if the quality is poor.

Level 3 has more information about the neighbors, or shows the error message when  you  have an incorrect command. Note that if there is no neighbor in the batman network, then it  will display nothing. Level 4 has so many information about the batman network, for example, how many packets  you sent, and sent to where, or how many packets you got, and received from where etc.  For example, you can run in normal start: \$ batmand  d 1 eth1  c connect via unix socket The batman daemon offers a unix socket interface to which you can connect. First, you have to create a batman daemon on your host, then use  c to connect to its  interface.
\$ batmand eth1 \$ batmand   c  d 1  In this case, you ask the daemon to output debug level 1 in your current shell. The batman  daemon will update the information after a period of time. Note that if you use  c flag, then you only can use  d to see the debug level.  b run connection in batch mode The debug information are updated  after a  period of time by default, so if you use  b it will  execute once and then stop.  \$ batmand eth1 \$ batmand   b  c  d 1  In this case, it means run debug level 1 once.
Note that  b can only be used with  c and debug level 1 \& 2.   g gateway class Gateway class can set eleven values, it means how much bandwidth you want to share.           default:            0  > this is not an internet gateway           allowed values:  1  > modem line                               2  > ISDN line                             3  > double ISDN                                4  > 256 KBit                               5  > UMTS / 0.5 MBit                             6  > 1 MBit                             7  > 2 MBit                             8  > 3 MBit                             9  > 5 MBit                            10  > 6 MBit                            11  > >6 Mbit You only can set the value in a normal start \$ batmand  g 7  d 3  eth1 Note that if you use debug level 3, then you will know whether you succeed setting the  gateway class or not.  r routing class Routing class can set to four values, it means this node in the batman network wants to  connect the Internet and chooses its internet gateway based on certain criteria:           default:           0  > set no default route           allowed values:  1  > use fast internet connection                              2  > use stable internet connection                              3  > use best statistic internet connection In level 1, B.A.T.M.A.N tries to find the best available connection by watching the uplinks  throughput and the link quality. In level 2, B.A.T.M.A.N observes the internet nodes and tries to find out which one is the  most reliable. This mode is not implemented yet but will follow in batman 0.3. In level 3, B.A.T.M.A.N only compares the link quality of the internet node and chooses the  one with the best connection.

\$ batmand  r 3  d 3  eth1

In this case, the batman daemon will choose the best statistic internet connection for you. Note that if you use debug level 3, then you will know whether you succeeded setting the  routing class or not.  p preferred gateway Set the default gateway  by yourself. Note that you have to use  r to tell batman daemon you want to set the default gateway,  because  r will be used if the preferred gateway is not available. \$ batmand  r 3  d 3  p 192.168.1.1 eth1 In this case, you set 192.168.1.1 as your preferred gateway, so all of your internet packets will  be sent to the 192.168.1.1.  a announce network(s) ? announce network ? means a node announces the connection to another network. For example, if you are the node A, and you can connect to ?other network?, then you can  execute  a to announce the gateway.

If the other nodes in the batman network want to connect to node B after receiving the  announce network information form node A, then they will know they can use node A as  gateway to reach node B. Now, you know what a announced network is, but executing this command is wrong: \$ batmand   a 192.168.100.101  eth1  Because you have to specify the netmask parameter and different netmask parameters cause  different results. Let's make a example: \$ batmand   a 192.168.100.101/32  eth1 In this case, it means that node A can only connect to node B, because your parameter is /32.
\$ batmand   a 192.168.100.101/24  eth1 In this case, it means that node A can connect to the whole 192.168.100.x network, because  your parameter is /24. So, if you use different netmask values, then the results are different.

Node A can  announce more than one network. To announce two networks execute the following  command: \$ batmand   a 192.168.100.101/24   a 192.168.200.201/24 eth1  Note that node A has to have a route to connect the node or network.  s visualization server Since no topology database is computed by the protocol an additional solution to create  topology graphs has been implemented, the Vis Server. Batman daemons may send their  local view about their single hop neighbors to the Vis server. The Vis Server collects the  information and provides data in a format similar to OLSR?s topology information output.  Therefore existing solutions to draw topology graphs developed for OLSR can be used to  visualize mesh clouds using B.A.T.M.A.N.

\section{Reference link}
? ?

http://open mesh.net/ <== B.A.T.M.A.N http://www.olsr.org/ <== OLSR 

\section{Troubleshooting}
\subsection{Why the batman daemon doesn't reload the setting after I fixed the main IP?}
You have to restart the batman daemon after you modified any network configuration,  otherwise the batman daemon won't use the new settings.

\$ killall batmand   \$ batmand eth1

\subsection{Why I can't connect to the Internet after setting the default gateway?}
You have to use NAT on your gateway or firewall if you use the  r or  p options to set default  route. \$ iptables  t nat  A POSTROUTING  o eth1  j MASQUERADE Note that you don't set the default route by yourself.

This work is licensed under the Creative Commons Attribution Noncommercial Share Alike 3.0 License. To 
view a copy of this license, visit http://creativecommons.org/licenses/by-nc-sa/3.0/ or send a letter to Creative  Commons, 559 Nathan Abbott Way, Stanford, California 94305, USA.
\end{document}
